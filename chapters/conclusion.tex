In this thesis, we introduced the fundamentals of \emph{reinforcement learning} -- a field of machine learning concerned with developing systems which learn based on interaction with the environment.
What distinguishes RL from other methods is the concept of reward signal, which provides feedback on whether the system behaves in a desirable way with regard to the goal of a task.

The transfer of knowledge from deep learning to the classical methods in RL is revolutionizing the field.
In the second part of Chapter \ref{chapter:background}, we presented some of the techniques that emerged to build more performant intelligent agents, leveraging the strength and robustness of neural networks, and especially of the ConvNet architecture.

In order to demonstrate the principles of deep RL in action, we have implemented a light framework, featuring a collection of algorithms in deep Q-learning, originally described in \cite{atari-dqn}, and its successors in the DQN family.
We benchmark and analyze the performance of our selected approaches on our task environment of choice -- Pac-Man. Solving Pac-Man is an interesting task and features some open challenges, as mentioned in Section \ref{section:modelling-the-problem}.

The following paragraphs give a summary of the most notable \emph{areas of potential improvement} for this work.

One possible extension to this thesis is hyperparemeter search.
\textbf{Hyperparameter tuning} or \textbf{search} is a problem in machine learning, which consists of tuning the set of hyperparameters to maximize the usefulness of the learning approach \cite{hyperparam-search-paper}.
Most studies have relied on the entire set of 49 Atari games to study the efficiency of their approaches.
In our thesis, however, we solve a more precisely defined task and could have benefited from tuning hyperparameters to some extent.
There is a strong probability that the set of hyperparameters which worked the best on Pong or Enduro does not work as well in Pac-Man.
However, one problem with auto-tuning is a practical problem, in that applying it to a relatively large problem such as ours requires a large amount of continuously-available compute resources.
On the basis of these large computation costs, we have decided that it would be cost ineffective.

Another such extension point is including policy-based and actor-critic methods into the benchmark.
In our background section, we touch on the subject of policy-based methods and their particular advantages.
The problem with policy-based RL is the novelty and increased complexity of the algorithms, over the more well-established techniques.
In other words, the state-of-the-art is not accessible to entry-level enthusiasts and would require far more work to implement from scratch, test and deploy.

To sum up, we have seen that autonomous RL agents can learn useful behaviour.
However, the study of autonomously learning RL agents in games has not been without challenges.
As we have seen with DQN and its successors, the method's greatest weakness remains sample efficiency -- agents need to train for millions of steps before they demonstrate any significant amount of ability.
And even then -- as we pay closer attention to generalization in deep RL -- we find that most Q-learning-based methods only manifest a somewhat limited ability to generalize, outside of very simple strategies (such as Pong, the game with the highest score obtained by DQN \cite{atari-dqn}).

With this in mind, the road ahead looks bright for RL, with a recently released paper from DeepMind – \emph{Agent57} (2020) \cite{agent57-paper}. We note that the latest research trends seem to be using mixed approaches: DQN and policy optimization algorithms, in an attempt to break the barriers set by their predecessors.
\emph{Agent57} has managed to break every record on the Atari benchmark and has obtained scores above the human baseline on 57 games in the set.