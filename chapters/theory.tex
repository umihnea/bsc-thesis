% General chapter description
In this chapter, we aim to present the central pieces of an agent-based approach in artificial intelligence: agent and environment.
We then explain how we can express elegant solutions to problems using this paradigm.
We introduce key concepts in the field of reinforcement learning (RL).
Diving deeper, we describe Q-learning and show how we can use neural networks to improve this classic algorithm.
Finally, we study successful applications of this approach.

\section{AI and Intelligent Agents}
% Link to brainstorming https://docs.google.com/document/d/1Dk2yzwWefNas-AZOo0nMAd4xbwd7LNU9qV0miYcRuBc/edit
% Here we talk about intelligent agents and environments and all the good stuff.

\section{Reinforcement Learning}
% Here we introduce many fundamental notions of reinforcement learning.
% Include a case study of AlphaGo

\section{Q-Learning}
% Here we start by describing naive Q-learning

% then we introduce deep Q-learning
% in this subsec, we will also do a quick introduction to neural networks

\section{Deep Q-Learning}

% Here we do a rundown of possible improvements of Q-learning: Double Q-learning and Duelling Q-learning

% Include a case study of DQN in the original Atari paper