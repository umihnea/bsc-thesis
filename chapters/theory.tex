% General chapter description
This chapter introduces the concepts necessary to understanding the subject of this thesis. We aim to:
\begin{itemize}
    \item present the central pieces of an \emph{agent-based approach} in artificial intelligence: agent and environment;
    \item explain how we can reason about problems using this paradigm;
    \item introduce key concepts in the field of \emph{reinforcement learning (RL)};
    \item describe \emph{Q-learning} and how we can use neural networks to improve the classic algorithm;
    \item exemplify a successful application of this approach.
\end{itemize}

\section{AI and Intelligent Agents}
% Link to brainstorming https://docs.google.com/document/d/1Dk2yzwWefNas-AZOo0nMAd4xbwd7LNU9qV0miYcRuBc/edit
An agent is defined as a {thing that acts} \cite{aima}.

\section{Reinforcement Learning}
% Here we introduce many fundamental notions of reinforcement learning
% Include a case study of AlphaGo

\section{Q-Learning}
% Here we start by describing naive Q-learning

% then we introduce deep Q-learning
% in this subsec, we will also do a quick introduction to neural networks

\section{Deep Q-Learning}

% Here we do a rundown of possible improvements of Q-learning: Double Q-learning and Duelling Q-learning

% Include a case study of DQN in the original Atari paper