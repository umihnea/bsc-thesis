\documentclass[12pt,twoside]{report}
\usepackage[utf8]{inputenc}
\usepackage{datetime}
\usepackage{amsmath,amssymb}
\usepackage{caption}
\usepackage{subcaption}
\usepackage[a4paper,width=150mm,top=25mm,bottom=25mm,bindingoffset=6mm]{geometry}
\setlength{\headheight}{15pt}

% hyperref
% source: timmurphy.org/2014/03/11/latex-table-of-contents-with-clickable-links/
\usepackage{color}
\usepackage{hyperref}
\hypersetup{
    colorlinks=false,
    linkcolor=blue,
    urlcolor=red,
    linktoc=all
}

% math commands
\newcommand\given[1][]{\:#1\vert\:}
\DeclareMathOperator*{\argmax}{\arg\!\max}

% fancyhdr
\usepackage{fancyhdr}
\newcommand{\changefont}{%
    \fontsize{9}{11}\selectfont
}
\pagestyle{fancy}
\fancyhead[LE,RO]{\changefont \slshape \rightmark} %section
\fancyhead[RE,LO]{\changefont \slshape \leftmark} %chapter

% graphicx
\usepackage{graphicx}
\graphicspath{{images/}}

% biblatex
\usepackage[square,numbers]{natbib}
\bibliographystyle{unsrtnat}

\usepackage[nottoc]{tocbibind}

% For blank pages
% solution detailed here https://tex.stackexchange.com/a/36881
\usepackage{afterpage}
\newcommand\blankpage{%
    \null
    \thispagestyle{empty}%
    \addtocounter{page}{-1}%
    \newpage}

\begin{document}

\begin{titlepage}
    \center % Center everything on the page

    % University, Faculty and Specialization
    {\scshape\LARGE Babeş-Bolyai University Cluj-Napoca \par}
    \vspace{0.125cm}
    {\scshape\LARGE Faculty of Mathematics and Computer Science\par}
    \vspace{0.125cm}
    {\scshape\LARGE Specialization Computer Science\par}
    \vspace{5cm}

    % Title section
    {
        \bfseries
        \LARGE \uppercase{Diploma Thesis} \\[1.5cm]
        \huge Using Deep Q-networks to learn Pac-Man
    }\\[4cm]

    % Author and Supervisor
    \begin{flushleft}
        \Large
            \textbf{Supervisor}
            \vspace{0.2cm}\\
        \Large
            Lect. Ph.D. Radu D. Găceanu
            \vspace{0.125cm}\\
    \end{flushleft}

    \begin{flushright}
        \Large
            \textbf{Author}
            \vspace{0.2cm}\\
        \Large
            Mihnea Ungureanu
    \end{flushright}

    \vfill

    % Year
    {\center \large 2020}
\end{titlepage}

% A blank page is required
% The blankpage custom command is defined above 
\blankpage

% Romanian title page
\begin{titlepage}
    \center % Center everything on the page
    
    % University, Faculty and Specialization
    {\scshape\LARGE Universitatea Babeş-Bolyai Cluj-Napoca \par}
    \vspace{0.125cm}
    {\scshape\LARGE Facultatea de Matematică şi Informatică \par}
    \vspace{0.125cm}
    {\scshape\LARGE Specializarea Informatică Engleză \par}
    \vspace{5cm}
    
    % Title section
    {
        \bfseries
        \LARGE LUCRARE DE LICENȚĂ \\[1.5cm]
        \huge Folosind Deep Q-networks pentru a învăța Pac-Man
    }\\[4cm]
    
    % Author and Supervisor
    \begin{flushleft}
        \Large
            \textbf{Conducător științific}
            \vspace{0.2cm}\\
        \Large
            Lect. dr. Radu D. Găceanu
            \vspace{0.125cm}\\
    \end{flushleft}
    
    \begin{flushright}
        \Large
            \textbf{Absolvent}
            \vspace{0.2cm}\\
        \Large
            Mihnea Ungureanu
    \end{flushright}
    
    \vfill
    
    % Year
    {\center \large 2020}
    
\end{titlepage}

\blankpage

\chapter*{Abstract}
This thesis presents an autonomous agent capable of learning optimal behaviour in video game environments.

We focus on a variant of the famous arcade game \emph{Pac-Man}, due to its high potential as a research tool in the study of intelligent agents, as well as intuitive appeal for human players.

Despite seeming like a toy problem, learning to solve virtual environments provides insight into complex, real-world environments at an insignificant fraction of the cost of running realistic simulations.

The main purpose of this thesis is to implement, test and analyze results of different algorithms for autonomous learning and select the most performant approach at playing \emph{Pac-Man}.
A secondary aim is to test whether Pac-Man is a strong benchmark environment in \textbf{reinforcement learning (RL)} and search for challenges which would allow researchers to gain further insight into AI agents.

The theoretical part of this thesis starts by exploring classical \textbf{reinforcement learning} techniques, then transitions to modern approaches which leverage \textbf{artificial neural networks (ANNs)} as functional approximators for RL.

The practical part consists of multiple algorithm implementations, as well as a light framework for training, evaluating and benchmarking the performance of RL agents.
The provided agents include a replication of DeepMind’s \textbf{deep  Q-learning} strategy \cite{atari-dqn}, as well as more performant successors to the original paper: DDQN \cite{ddqn-paper} and Prioritized Experience Replay \cite{per-paper}.

The agents are implemented using PyTorch, a powerful Python machine learning library based on Torch. We use the Atari toolkit from OpenAI Gym, which was designed for RL and provides interfaces to keep track of in-game score, lives and other indicators.
Finally, we analyze our results using Jupyter Notebook, a well-established Python data science tool.

\hfill \break
This work is the result of my own activity. I have neither given nor received unauthorised assistance on this work.
\begin{flushright}
    Mihnea Ungureanu
\end{flushright}


% ToC placed between abstract and introduction
\tableofcontents


\chapter{Introduction}
Over the relatively recent years, \textbf{reinforcement learning (RL)} has gained immense popularity.
Stemming from the agent-based approach to artificial intelligence, it is a unique paradigm within the field of machine learning.
RL directly focuses on building agents that make optimal decisions in an environment in order to maximize a notion of cumulative reward.
RL has a broad focus and researchers have trained RL agents to solve a variety of tasks.

One of the field's most notable achievments is DeepMind's \textbf{AlphaGo} family of Go-playing programs \cite{ago, alpha-zero}.
Beating humans at Go was widely acknowledged as the next frontier in game-playing AI after the defeat of Gary Kasparov in 1997\footnotemark.
Go was a natural progression from chess due to its larger state space and higher branching factor.
In 2016, AlphaGo became world renowned after defeating Lee Sedol, 18-times world champion, thus becoming the first computer system achieving \textbf{superhuman perfomance} in the game of Go.
\footnotetext{Gary Kasparov, the legendary chess champion, was defeated by IBM chess-playing computer DeepBlue in 1997, marking the first superhuman computer performance in chess.}

However, a multitude of interesting studies in RL focuses on agents learning in \textbf{video games} or other game-like environments.
In Mnih et al.'s Atari DQN paper \cite{atari-dqn}, which is covered in this thesis, the DeepMind team proposes a method to train agents to play Atari video games exclusively from raw video input.
The study produced agents able to surpass human-level performance in Breakout, Enduro and Pong.
More importantly, it inspired a long chain of attempts of the research community to improve the original algorithm and beat the records it set.
% Starcraft II and Dota 2

This thesis aims to explore a challenging video game environment by analyzing the performance of existing algorithms in order to discover potentially underexplored areas in the design of AI agents.
We hope to outline some limitations of the current standards in order to benefit the further development of this field.

Toy problems such as playing video games from raw sensory input are the training ground which precedes real-world environments which require real-time intelligent control, such as self-driving cars.
Innovation in AI, as in all domains, begins at a small scale.

In order to accomplish our objective, we aim to build an agent framework that provides intuitive yet easily extensible tools, a solid testbed for experiments and means of data analysis.

Our choice for an environment for studying agents is a variant of popular arcade video game \emph{Pac-Man}, developed and released by Namco Ltd. in 1980.
This choice is motivated by two factors: the potential challenges it poses to AI and its fun and intuitive appeal for humans.
The game is challenging for AI agents because it involves balancing the goal of exploration and point acquisition with escaping different classes of enemies that chase the player.

This thesis can be roughly divided into a theoretical part which is compact and contained entirely within one exhaustive chapter, and a practical part, composed of an application chapter followed by an analysis of the obtained results.

\emph{Chapter \ref{chapter:background}, ``Background''} presents a comprehensive overview of the theory necessary to understanding reinforcement learning.
We start by introducing agent-based AI, then continue with a
section dedicated to understanding fundamental concepts in RL.
The second half of the chapter is dedicated to deep reinforcement learning, which can roughly be defined as combination of classic RL and neural networks.
Among the topics covered in the second half are convolutional neural networks (ConvNets) and some proeminent solution methods in deep RL.

\emph{Chapter \ref{chapter:practical}, ``Playing Pac-Man using Deep Q-learning''} states the problem and presents the chosen algorithms for our agents.
In particular, we present deep Q-networks (DQN) and two of its derviatives -- double DQN and prioritized experience replay.
Furthermore, we go in-depth over the structure of the framework and its implementation.

\emph{Chapter \ref{chapter:results}, ``Results and Evaluation''} presents our methods of measuring the performance of the agents and analyzes the obtained data of our various experiments.

In our ending chapter, \emph{Chapter \ref{chapter:conclusion}, ``Conclustion and Future Work''}, we sum up our methods and results and we identify future areas of improvement. Additionally, we try to predict the direction of RL research based on our conclusions.

\chapter{Background} \label{chapter:background}
This chapter introduces the concepts necessary to understanding the subject of this thesis -- implementing a learning agent using reinforcement learning.

In \emph{Chapter \ref{agents-intro}. Intelligent Agents} we aim to present the central pieces of the \textbf{agent-based approach} in artificial intelligence: agent and environment.
We explore the constituents of a learning agent and explain how we can reason about problems using the notion of task environment.

\emph{Chapter \ref{reinforcement-learning}. Reinforcement Learning} introduces key concepts in the field of RL. This chapter deals with \textbf{theoretical fundamentals} (Markov decision processes) and their application in solving problems in this paradigm.

In the same chapter, we present an array of \textbf{classical solution methods}, ranging from Monte Carlo (which simply applies statistical sampling to learn from an unknown environment) to more subtle methods such as temporal-difference (TD) learning.

In \emph{Chapter \ref{deep-rl}. Deep Reinforcement Learning}, we explain how to scale the classical solutions to solve larger (and more relevant) problems using \textbf{functional approximation}.
Deep RL represents an \textbf{active area of reasearch} and an extensive survey of the state-of-the-art would be out of the scope of this paper.
Instead, we cover a few tried-and-tested solution methods which had moderate success in their application.

\clearpage

\section{Intelligent Agents} \label{agents-intro}
This section aims to provide a short introduction to the central pieces of an \emph{agent-based approach} in artificial intelligence: agent and environment.
Understanding how to formulate relevant problems in terms of agents learning from controlled environments is necessary to reinforcement learning, the core technique presented in this thesis.

The concept of rational agent is central to the study of artificial intelligence. 
It is a natural evolution from the \emph{laws of thought} approach in AI (i.e., using a set of logical rules to derive new knowledge).

As pointed out in \cite{aima}, inference does not capture all of rationality.
As complex environments imply \textit{uncertain situations} (where there is no provably correct choice to make), a finite set of rules can prove inadequate.
The agent-based approach captures a more complete definition of rationality by shifting focus from making correct inferrences to producing \emph{efficient behaviour}.

\subsection{Agent Function and Agent Program}
According to \cite{aima}, an \textbf{agent} is simply something that percieves and acts in an environment.
An agent can be split into two major components, each fulfilling a fundamental function:
\emph{perceptors} (sensors) manage \emph{perception} and \emph{actuators} act upon the environment.
This is the simplest useful division. We present a more elaborate division later, in Section \ref{learning-agents}.

It is sometimes useful to reason about agents from a mathematical perspective.
For this, there exists the concept of \emph{agent function}.
An \textbf{agent function} maps what the agent percieves (every ``snapshot'' of perception, called a \emph{percept}) to an action.
Outside the world of very small, ideal examples, any agent function can be considered \emph{infinite}.
The implementation of an agent function is an \textbf{agent program}, which is what AI is actually concerned with building and perfecting.
This implementation is, of course, finite.

The term \emph{agent} \textbf{can} designate physical robots working in real-life environments.
However, in this thesis we use the term \textit{agent} to refer exclusively to softbots (or software robots).
The problem that we solve can be entirely modeled in a virtual space.

\subsection{Task Environments}
Simply put, \textbf{task environments} are a way to express a problem to which an agent is the solution.
The design of the agent depends on the attributes of its environment.

In this paradigm, we can decompose a task into several subcomponents, often abbreviated \textbf{PEAS} \cite{aima}, comprising of:
\begin{enumerate}
    \item \textbf{performance measure}, a function which measures the performance of the agent;
    \item \textbf{environment}, the ``world'' the agents interacts with;
    \item \textbf{actuators}, the agent's means of action;
    \item \textbf{sensors}, the agent's means of perception;
\end{enumerate}

The \textbf{performance measure} has many possible representations, depending on the problem.
Its primary purpose is to measure how effective the behaviour of the agent is, with regard to the given goals.
In reinforcement learning, we measure this effectiveness using a \textbf{reward function}.
We will elaborate on this in our section on reinforcement learning (Section \ref{reinforcement-learning}).

\subsubsection{Environment Classification}

Modeling real-world use cases requires understanding the vast diversity of environments.
In the following paragraphs, we present some \textbf{classification criteria} for task environments that are also provided in \cite{aima}.

\textbf{Observability} can be \emph{full} or \emph{partial}.
In a fully observable environment, the agent has full information of all the relevant aspects of the world (for example, a chess-playing agent sees the entire chessboard).
The latter means the agent operates with restricted information.
Many real-world problems have partial observability due to sensor noise or computational constraints (i.e., an upper limit on memory and processing power).

\textbf{Single- or multi-agent}.
Multi-agent environments are a tool for studying emergent behaviour, wherein relatively simple agents construct elegant and complex solutions to problems.
The techiques concerning the design of multi-agent systems are frequently subtle and can be treated as another subject completely.

\textbf{Stochasticity}.
In a \emph{deterministic} environment, every state is completely determined by its previous. Otherwise, the environment is \emph{stochastic}. An environment is \textbf{uncertain} when it is stochastic and/or partially observable.

\textbf{Discreteness} and \textbf{continuity} describe action spaces and the way time passes inside the environment.
This is best understood by example.
A chess-playing agent has a finite number of possible actions at every step.
Chess is therefore a \emph{discrete} environment.
An agent driving a car, on the other hand, has control over some real-valued parameters of the car (steering angle, speed and so on).
The domain of possible actions in real-world driving is therefore \emph{continuous}.
In the same way, we can describe time: in discrete time, time spent deciding does not influence the outcome. The opposite is true for continuous time representation -- in real-time driving, every second spent not making a decision is stalling.

\textbf{Known} and \textbf{unknown}.
This refers to the agent's knowledge of the laws governing its environment. In a known environment, the agent knows the rules of its world a priori. Otherwise, an agent has to learn how the environment works in order to make effective decisions.

% Can include environment classes and environment generators

\subsection{Learning Agents} \label{learning-agents}
In this thesis, we are intereseted in creating an autonomous learning agent.
An agent system would be useless by modern standards if it had no capacity to \textbf{learn from past experience}.
In this section, we aim to understand learning agents by going through the anatomy of their system.
It is worth mentioning that the actual architecture of the program will not be a one-to-one match to this conceptual tour.

Russel and Norvig provide an excellent description of \textbf{learning} in intelligent agents in \cite{aima}:
\begin{quote}
    Learning in intelligent agents can be summarized as a process of modification of each component of the agent to bring the components into closer agreement with the available feedback information, thereby improving the overall performance of the agent.
\end{quote}

In our introduction, we said that agents have an advantage over knowledge-based approaches.
This advantage comes from the fact that agents can learn in initially unknown environments.
Learning allows agents expand on initial knowledge, if any.
Learning agent approaches can achieve substantial results starting with \textbf{zero expert knowledge}, as demonstrated by the Go agent in AlphaZero \cite{alpha-zero}.

According to \cite{aima}, an agent can be divided in four subsystems:
\begin{enumerate}
    \item \textbf{The learning element} is resposible for improving the agent by incorporating information from the environment.
    \item \textbf{The performance element} perceives the environment and selects actions and is the object of improvement.
    \item \textbf{The critic} provides feedback to the learning element.
    \item \textbf{The problem generator} suggests novel actions, in order to expand knowledge.    
\end{enumerate}

A learning agent overview figure, closely following the schematic provided in \cite{aima}, is shown in Figure \ref{fig:aima-learning-agent}.

\begin{figure}[ht]
    \includegraphics[width=0.7\textwidth]{aima-general-learning-agent}
    \centering
    \caption{An overview schematic for the generalized model of a learning agent.}
    \label{fig:aima-learning-agent}
\end{figure}

\subsubsection{Learning and Performance}
The performance element fulfills the basic functions of the agent -- it perceives, decides and acts.
The learning element's job is to improve this performance based on feedback, coming from an outside observer (the critic).
The learning element's design varies depending on the problem and how the agent is meant to interact with it.

\subsubsection{Critics}
An agent cannot tell whether its behaviour is effective just by observing changes in the environment.
It needs a \textbf{critic} that provides feedback to the learning element about its performance in the world.
The critic can esentially be viewed as a stand-alone, external entity.
It is based on a fixed performance standard which does not depend on the agent's internal state.
In fact, it is imperative for the critic to be kept separate frim the agent state, to prevent the agent of ``deluding'' itself that it is doing a good job \cite{aima}.

\subsubsection{Problem Generators}
Agents can get stuck making decisions that yield good but suboptimal results.
This is known as the problem of \emph{local optima}, a well-known problem in mathematical optimization which is very relevent in AI.
This happens because the agent bases its calculations on limited knowledge.
It needs a component to take care of adding new information into its system.
The problem generator ensures effective \textbf{exploration}:
choosing options that yield suboptimal results in the short term, in order to discover better strategies on the long term.

\clearpage  % pushes the RL chapter onto the next page instead of having dangling

\section{Reinforcement Learning} \label{reinforcement-learning}
\textbf{Reinforcement learning (RL)} is an area of machine learning that epitomizes the agent-based approach.
RL is unique within its field as it is directly focused on goal-directed learning from agent-environment interaction \cite{rlai}.
At the core of RL lays the reward function (or signal), which perfectly describes goals established by the problem.

In the following chapter, we start by summarizing finite MDPs -- the mathematical underpinning for reinforcement learning, while also familiarizing with the the notation used throughout this paper.

In section \ref{rl:dp} we cover the first solution technique: using dynamic programming in a known, finite MDP to find its optimal solution.
In section \ref{rl:mc}, we generalize this method to solve unknown MDPs -- the complete formulation of a RL problem -- using Monte-Carlo learning.
In section \ref{rl:td}, we visit temporal-difference learning and finally develop our solution using Q-learning in section \ref{rl:q-learning}.

\subsection{Components and Notation}
Introduction into reinforcement learning supposes familiarity with the paradigm's simple-but-effective mathematical framework.
We present the theory behind Markov decision processes (MDPs) and then we shift our attention to explaing the components of a reinforcement learning agent.

\subsubsection{Finite Markov Decision Processes}
Finite Markov decision processes are a formalization of a reinforcement learning problem -- any problem which can be represented as a finite MDP can be solved using a RL technique.

The feedback loop is central to understanding the reinforcement learning dynamic.
This is just a reiteration of the contents in our previous chapter:
an \textbf{agent} acts upon an \textbf{environment}, which reacts according to its set of governing rules.
Each iteration takes place at a discrete time step \(t = 0,1,2,\dots \) (although there are continuous variants of a Markov process, we are not concerned with them).

\begin{figure}[ht]
    \centering
    \caption{Feedback Loop. (Partial reproduction from \cite{rlai})}
    \vspace*{0.2cm}
    \includegraphics[width=0.5\textwidth]{agent-env-fig}
\end{figure}

\begin{enumerate}
    \item The agent is in state \(s\) and picks action \(a\).
    \item It receives a reward \(r\) for its action. This (immediate) reward signal is problem-defined and quantifies the agent’s goal.
    \item The environment reacts, sending the agent into the next state \(s'\).
\end{enumerate}

A \textbf{Markov decision process (MDP)} is defined as a tuple \(\langle S, A, P, R, \gamma \rangle\), in which:
\begin{enumerate}
    \item \(S\) is the finite set of all states.
    \item \(A\) is the finite set of all possible actions.
    \item \(P\) is the state transition probability function, where \(P_a(s, s')\) denotes the probability of starting in state \(s\) and ending up in state \(s'\) by taking action \(a\).
    \item \(R\) is the reward function, where \(R_a(s, s')\) is a value in \(\mathbb{R}\) and represents the reward received for starting in state \(s\) and ending up in state \(s'\) by taking action \(a\).
    \item \(\gamma\) is the discount factor (used when computing the \textbf{return}), \(\gamma \in [0, 1]\).
\end{enumerate}

The \textbf{state} refers to the internal state of the agent \footnote{Sometimes we may also refer to the state of the environment, but context will clarify. Environment state and agent state only completely match in full observability environments.}.
A state \(S_t\) contains relevant information from the environment at a given time step \(t\).
A key point in Markov processes is that a correctly formulated state captures all the relevant information and removes the need of explicitly memorizing state history.
This is known as the \textbf{Markov property} \cite{silver-lectures}.

A \textbf{policy \(\pi\)} completely characterizes an agent’s behaviour.
``It is a mapping from perceived states of the environment to actions to be taken in those states'' \cite{rlai}.
Policies can be deterministic (i.e. ``if this then that'' rules) or stochastic.
A \textbf{stochastic policy}, denoted \(\pi(a \given s)\), is a probability distribution over actions, given a state.

The \textbf{reward} models the problem-defined goals as a scalar that can be associated with each state transition.
``The reward signal is the primary basis of altering the policy.'' \cite{rlai}.
The assertion that we can completely and correctly model all goals using reward functions is central to the field of RL.
This is called the \textbf{Reward Hypothesis} and is formulated below:
\begin{quotation}
    That all of what we mean by goals and purposes can be well thought of as maximization of the expected value of the cumulative sum of a received scalar signal (called reward). \textit{(from \cite{rlai}, Chapter 3.2)}
\end{quotation}

The \textbf{return} \(G_t\) is the \textbf{cumulative reward} obtained by the agent starting from time step \(t\).
This is what a reinforcement learning system is supposed to maximize.
There are multiple mathematical models used to represent the return.
In equation \ref{eqn:return}, we use an infinite-horizon sum with discounting.

\begin{equation} \label{eqn:return}
    G_t = R_{t+1} + \gamma R_{t+2} + \dots = \sum_{k = 0}^{\infty}{\gamma^{k} R_{t + k + 1}}    
\end{equation}

Discounting is, intuitively, a way to control how much the agent cares about future rewards.
More on this topic can be found in either theoretical reference \cite{rlai, silver-lectures}.

% todo: explain why discounting is used and why it is convenient

\subsubsection{Value Functions} \label{rl:value-func}
The \textbf{state-value function} for policy \(\pi\) is denoted by \(V^{\pi}(s)\) and measures how good each state is, with regard to the long-term potential of that state.
Whereas rewards characterize the immediate value of a state, the value function measures its long-term desirability \cite{rlai}, under the given policy.

\begin{equation} \label{eqn:value}
    V^{\pi}(s) = \mathbb{E}_{\pi}\{ G_t \given S_t = s \}
\end{equation}

In Equation \ref{eqn:value}, \(\mathbb{E}\) represents the expected value of a random variable. Here we are interested in the expected value of the return, or simply -- the \emph{expected return}.
By \(\mathbb{E}_{\pi}\) we denote the expected value conditional on following policy \(\pi\).
We will maintain this notation in the expressions below.

The \textbf{action-value function} \(Q^{\pi}(s, a)\), also known as the Q-value function, measures the desirability of a state-action pair.
More explicitly, it gives the expected return of starting in state $s$ and taking action $a$, then continuing to follow policy $\pi$.

\begin{equation}
    Q^{\pi}(s, a) = \mathbb{E}_{\pi}\{ G_t \given S_t = s, A_t = a \}
\end{equation}

Action-value functions are especially important in \textbf{model-free methods} (see \emph{Models} in Section \ref{rl:model}), which are predominantly the focus of this paper.
In model-based algorithms, the value function $v$ can uniquely determine a policy: we can simply pick the best reward-state pair.
This naturally requires knowing the full dynamics of the MDP (knowing all successor states of $s$).

However, for many tasks, especially those relevant in the real-world, it is (at least) impractical to know or to model the environment.
In contrast, keeping track of both states \emph{and} actions allows model-free learning algorithms to construct policies without knowledge of environment dynamics.

\subsubsection{Optimal Value Functions}

There are optimal variants of the same value functions.
In order to understand the concept of \textbf{optimality}, we need to define an \textbf{ordering} between policies.
Considering $\pi$ and $\pi'$, we say that $\pi' \geq \pi$ if $V^{\pi'}(s) \geq V^{\pi}(s)$ for every $s \in S$.
Using this definition, we can say that there exists a $\pi^{\star}$ which is the \textbf{optimal policy}, being greater than or equal to all other possible policies.

The \textbf{optimal value function $V^*(s)$} represents the maximum value obtained in state $s$, over all policies, as briefly summed up the relation below.

\begin{equation} \label{eqn:opt-value-fun}
    V^*(s) = \max_{\pi}{V^{\pi}(s)}
\end{equation}

Similarly, \textbf{optimal action-value function $Q^*(s, a)$} is the maximum value of being in state $s$ and taking action $a$, considered over all policies.

\begin{equation} \label{eqn:opt-Q-fun}
    Q^*(s) = \max_{\pi}{Q^{\pi}(s, a)}
\end{equation}

Value function approximation is the foundation of multiple solution methods in RL.
A simple example can be seen in Monte Carlo methods, where the value of a state \(s\) is approximated by averaging over multiple trajectories starting from \(s\).

\subsubsection{Models} \label{rl:model}
A \textbf{model} (of the environment) allows the agent to plan and make predictions of its environment.
Some algorithms focus explicitly on learning a model and use it for \textbf{planning}.
This approach is called \textbf{model-based}.
In a model-based method, an agent can query the model to simulate what would happen with the environment before actually choosing an action.
Approaches without a model are called \textbf{model-free}. Model-free agents are ``explicitly trial-and-error learners'' \cite{rlai}.

\begin{figure}[ht]
    \caption{A way of classifying RL methods based on whether they have a value function, policy or model. (Reproduced from David Silver's lectures. \cite{silver-lectures})}
    \vspace*{0.2cm}
    \centering
    \includegraphics[width=0.5\textwidth]{silver-venn}
\end{figure}

\subsection{Bellman Equations} \label{rl:bellman}
The \textbf{Bellman equations} are a set of relations conceived by Richard Bellman in the 1950s, in the context of solving optimal control using dynamic programming.
In the reinforcement learning context, they interrelate the state-value and action-value functions and provide the basis of solving an MDP whether known or unknown.

Equation \ref{eqn:bellman-expectation} states the \textbf{Bellman expectation equation}.
The key idea behind this class of Bellman equations is that they express the value of a state as expected sum of immediate reward of being in that state,  plus discounted value of successor state.
This transformation becomes clear if we use our return-based exprimation of value function $V^{\pi}(s)$ in Equation \ref{eqn:value} and reorganize its terms:

\begin{equation} \label{eqn:bellman-expectation}
\begin{aligned}
    V^{\pi}(s)
        &= \mathbb{E}_{\pi}\{ G_t \given S_t = s \} \\
        &= \mathbb{E}_{\pi}\{ R_{t+1} + \gamma R_{t+2} + \dots \given S_t = s \} \\
        &= \mathbb{E}_{\pi}\{ R_{t+1} + \gamma V^{\pi}(S_{t+1}) \given S_{t} = s \} \\
\end{aligned}
\end{equation}

In Figure \ref{fig:expectation-backup}, we model the equation using a backup tree\footnote{Called a backup or update tree, it visualizes the way that state values are updated. This visualization convention is proposed in and used throughout \cite{rlai}.}, where each open circle represents a state, while closed cicles represent state-action pairs.

In state $s$, the agent has a set of available actions according to its policy $\pi$.
After the agent chooses action $a$, the environment responds with reward $r$ and throws the agent into the next state  $s'$ -- one of multiple possible states according to environment dynamics.

Equation \ref{fig:expectation-backup} computes the value of state $s$ by \emph{averaging} over all possible trajectories, \emph{weighing} each by its probability of occuring.

\begin{figure}[ht]
    \centering
    \caption{Bellman expectation backup.} \label{fig:expectation-backup}
    \vspace*{0.2cm}
    \includegraphics[width=0.4\textwidth]{bellman-expectation-backup}
\end{figure}

In cases in which the MDP is \emph{known}, we can leverage the fact that we know the transition probabilities $P$ to \emph{directly compute} the expectation in \ref{eqn:bellman-expectation}.
\begin{equation}
    \begin{aligned}
        V^{\pi}(s)
            &= \mathbb{E}_{\pi}\{ R_{t+1} + \gamma V^{\pi}(S_{t+1}) \given S_{t} = s \} \\
            &= \sum_{a \in A}{
                \pi(a \given s) ( R(s, a) + \gamma \sum_{s' \in S}{
                    P_a(s, s') V_{\pi}(s')
                })
            }
    \end{aligned}
\end{equation}

There exists another important relation with important applications in RL -- the \textbf{Bellman optimality equation}.
One way of expressing the optimality equation, provided in \ref{eqn:bellman-optimality}, simply states the fact that the optimal value of a state must be the value of the best possible action from that state \cite{rlai}.
This is an example of interdependence of the two value functions, although it is important to mention that we can express $q_{*}$ in terms of itself as well.

\begin{equation} \label{eqn:bellman-optimality}
    \begin{aligned}
        v_*(s) = \max_{a \in A} q_{*}(s, a)
    \end{aligned}
\end{equation}

The optimality equation is non-linear, but we cover approximating solutions using iterative methods in our dynamic programming section at \ref{rl:dp}, specifically to solve planning problems (which require finding the optimal policy).

\subsection{Dynamic Programming} \label{rl:dp}
\textbf{Dynamic programming (DP)} methods are an important theoretical starting point for reinforcement learning methods. DP problems require full knowledge of an MDP and use general methods to compute optimal solutions.

While being theoretically useful, there are key bottlenecks which prevent DP from achieving practicality:

\begin{enumerate}
    \item DP represents a \textbf{subset} of a reinforcement learning problem, as it requires full knowledge of an MDP and, thus, is not suitable in unknown environments.
    \item DP is \textbf{computationally expensive} as it searches for optimal solutions over the entire state space (suffers from the ``curse of dimensionality'' \cite{rlai}).
    \item DP cannot be applied with \textbf{continuous} spaces, unless problems satisfy additional criteria \cite{rlai}.
\end{enumerate}

There are two key problems that can be solved using DP in a fully known MDP using the Bellman equations.

\textbf{Policy evaluation}, also reffered to as \emph{prediction}, starts with a given policy $\pi$ and is concerned with finding its value function $v_{\pi}$.
In theory, the value of each state can be computed by solving a system of Bellman equations, one for each state.
However, the computation required is impractical for most problems.
A more suitable method is \emph{iterative policy evaluation}, which uses equation \ref{eqn:bellman-expectation} iteratively, starting from an arbitraty value function $v_0$ and eventually converging to $v_{\pi}$.

\textbf{Planning} is, in constrast, concerned with finding the optimal policy $\pi^{\star}$ starting from an arbitraty point in the space.

Planning can be done using \emph{policy iteration}.
At every iteration $k$, this method evaluates the given policy $\pi$ to find its value function $v_{\pi}$ (solving the above problem).
Then, it constructs a new policy $\pi'$ which acts \emph{greedily} with regard to $v_{\pi}$.
This is proven to eventually converge to $\pi^*$ as $k \to \infty$ \cite{rlai}.

\begin{figure}[h]
    \caption{An illustration of policy iteration. (From David Silver's lectures. \cite{silver-lectures})}
    \centering
    \includegraphics[width=0.4\textwidth]{silver-policy-iteration}
\end{figure}

Using policy iteration can be extremely expensive, as the method adds an additional layer of computation on top of already costly policy evalutation.
The idea of \textbf{value iteration} comes from collapsing the evaluation step with the policy construction step, by applying ``one sweep'' policy iteration -- only updating each value once at every step.

\subsection{Monte-Carlo Learning} \label{rl:mc}


\subsection{Temporal-Difference Learning} \label{rl:td}
TD is a class of solution methods in RL that is part of the model-free subset \cite{rlai}.
An advantage of TD learning can learn from incomplete episodes \cite{long-peak-rl}.

\subsection{Q-Learning} \label{rl:q-learning}
Q-learning \cite{Watkins1992} appeared in 1992 and is considered one of the early breakthroughs of RL \cite{rlai}.
It is a simple algorithm allowing an agent to learn an optimal policy in an unknown environment.

\subsection{Functional Approximators}

\section{Deep Reinforcement Learning} \label{deep-rl}
Researchers found that advances in the field of deep learning -- a broad family of methods based on artificial neural networks -- can be used to enhance the classical RL algorithms.
This led to the expansion of the field, marking the birth of \textbf{deep reinforcement learning (deep RL)}.

Classic RL assumes that the state space is finite and can be modeled in a tabular manner.
However, we seldom encounter real-world, interesting problems where this assumption holds.
Deep RL is an open field striving to solve \emph{more complex} problems than classical methods allow.
This is done through the ability of functional approximators to model high-dimensional spaces.

The field has recently surged, after a series of new algorithms and succesful applications were published.
In this chapter, we try to give an overview of the main methods of learning but mainly focus on developing \emph{Mnih et al.'s \textbf{DQN} (2013)} \cite{atari-dqn}.

In Section \ref{section:convnets}, we present \textbf{convolutional neural networks}, a class of artificial neural networks which became the foundation of multiple algorithms in deep RL.

Section \ref{section:dqn} covers using neural networks (called deep Q-networks in the paper \cite{atari-dqn}) as approximators in Q-learning in order to build autonomous agents for complex model-free environments.

The rest of the chapter (Sections \ref{section:policy-opt} and \ref{section:actor-critic}) deals with alternative strategies for doing (deep) model-free control, namely \textbf{policy optimization} and \textbf{actor-critic} methods respectively.

\clearpage

\subsection{Convolutional Neural Networks (ConvNets)} \label{section:convnets}
\textbf{Convolutional neural networks} (often shortened as ConvNets or CNNs) are a specific type of artificial neural network for processing data that has a known grid-like topology \cite{Goodfellow-et-al-2016}.

ConvNets exploit the \textbf{spatial nature} of the data.
This property makes them especially useful over a vast spectrum of problems, such as image classification, audio signal processing, analyzing financial time-series etc.
Among those, image classification tasks are the primary reason for the architecture's popularity.

The \textbf{origin} of convolutional networks can be traced back to Kunihiko Fukushima's \emph{neocognitron} (1979).
Kunihiko’s design \cite{neocognitron-paper} was itself inspired by earlier breakthroughs in neuroscience -- namely studies of the visual cortex of mammals (Hubel \& Wiesel, 1959).
LeCun et al. (1989) further improved this model by introducting backpropagation training, which set the standard for today’s architectures.

\subsubsection{Overview}
A convolutional network contains at least one convolutional layer in its structure.
A typical convolutional layer (represented in Figure \ref{fig:conv-layer}) is composed of three stages \cite{Goodfellow-et-al-2016}:
\begin{enumerate}
    \item The \textbf{convolution stage} convolves the input data with a filter (kernel), which results in a set of linear activations.
    The kernel is the learned element, i.e. we learn the weights of the kernel as we would learn the weights of a linear layer.
    \item The \textbf{detector stage} pipes each linear activation from the convolution layer thorugh a non-linear activation function (e.g. ReLU \footnotemark)
    \item The \textbf{pooling stage} is an optional stage which reduces the output size by computing summary statistics over the data.
\end{enumerate}
\footnotetext{standing for \emph{rectified linear unit}, it is commonly used as an activation function in neural networks. In its classic form, it can be expressed as $f(x) = max(0, x)$.}

\begin{figure}[h]
    \centering
    \includegraphics[width=0.7\textwidth]{conv-layer-structure.png}
    \caption{A schematic of a convolutional layer.}
    \label{fig:conv-layer}
\end{figure}

In this section, we consider the \textbf{input} and \textbf{output data} to be 3D tensors, consisting of a spatial dimension (width and height) and a depth dimension.
Visualizing this process with images in mind is particularly helpful to understanding.
The spatial dimensions map exactly to the width and height of an image, and each color channel can be considered a separate depth level.


\subsubsection{The Convolutional Stage}
This stage performs the convolution operation which is the most computationally demanding procedure of the layer.
The output from this stage is a \textbf{feature map} of the input, which is then passed as linear activation to the detector stage, likewise to fully-connected feedforward networks.

This \textbf{operation} consists of sliding a \emph{kernel} over the input.
At every step, we compute the dot product between the kernel and the volume of input it currently overlaps.
The \textbf{kernel} (or filter) is a tensor of adjustable weights, which is small relative to the image's spatial dimentions but must cover the entire depth of the image.
The effects of the convolutional stage are determined by a number of hyperparameters: depth, stride and padding.

A convolutional stage can learn multiple independent kernels.
The number of kernels is given by the \textbf{depth} hyperparameter.
It is conveniently called such because it corresponds to the depth of the output of this stage.
A \textbf{depth column} (or fibre) \cite{stanford-convnets} is a set of neurons ``looking'' to the same region in the input (where each neuron corresponds to a different kernel) .

The \textbf{stride} specifies the distance which we slide the kernel with over the input at each step.
Some models provide separate stride parameters for each axis (horizontal and vertical) \cite{Goodfellow-et-al-2016}.

\textbf{Zero-padding} (or simply padding) enables padding our input data with zeros along the borders. It is useful for controlling output size, especially in cases where we want to preserve the input size over multiple layers.

\begin{figure}[h]
    \centering
    \includegraphics[width=0.6\textwidth]{conv-stride-padding.png}
    \caption{(Left) Using a stride of 1 and a zero-padding of 1 to preserve the original spatial dimensions of the input. (Right) Using a padding of 1 and a stride of 2, which results in fewer output values. (Illustration from \cite{stanford-convnets})}
    \label{fig:conv-stride-padding}
\end{figure}
 
The properties of convolution confer ConvNets a myriad of advantages over fully-connected feedforward neural networks.
This gain in efficiency comes from the assumption that the data has an underlying spatial structure, which can be exploited better by ConvNets than by alternatives.

Typical fully-connected layers relate all their outputs to all their inputs by separate, \emph{independently adjustable} weights.
This requires storing weight matrices as large as the input data itself, rising the computational cost as input size grows.
In contrast, kernels are significantly smaller than the input volume is (in the spatial dimensions).
Kernels are also reused across the entire layer (\textbf{parameter sharing}), resulting in smaller memory requirements.

ConvNets feature \textbf{sparse interactions}, in contrast to the dense interactions of fully-connected layers in which every output is related by a weight to each of the input values.
This boosts learning efficiency by reducing the number of necessary computations.

Learning smaller, shared kernels can be a means of controlling \emph{overfitting} \cite{Goodfellow-et-al-2016}. Moreover, ConvNets can operate on input data of \textbf{varying sizes} without making adjustments to model architecture.


\subsubsection{Pooling}
The \textbf{pooling stage} is an optional stage in the ConvNet architecture which performs a \emph{downsampling} of its inputs.
The role of this stage is to gradually reduce computation (and memory requirements) in the network by reducing the spatial dimension of the data \cite{stanford-convnets}.
It processes output from the convolutional stage by applying a pooling function.

A \textbf{pooling function} computes a summary statistic (e.g. maximum, average, $l^2$-norm etc.) over a neighbourhood of input values \cite{Goodfellow-et-al-2016}.
The pooling stage has a similar operation to the convolutional stage:
a moving window is slided over and across the input data and at each step, the summary is computed.

Each depth slice of the data is processed independently.
An example for a simple but commonly used operation (max pooling) is shown in Figure \ref{fig:maxpooling}.

\begin{figure}[h]
    \centering
    \includegraphics[width=0.8\textwidth]{conv-maxpooling.jpg}
    \caption{Simple example illustrating max pooling (on a single depth level). Illustration from \cite{stanford-convnets}.}
    \label{fig:maxpooling}
\end{figure}

\subsection{Deep Q-Networks (DQN)} \label{section:dqn}
To reiterate our chapter introduction, a key challenge of applying RL to real-world problems is data representation (in the form of policies and value functions).
Real applications require modelling large, high-dimensional problem spaces and hence depend upon functional approximation.

However, functional approximators, whether linear or nonlinear, require ready access to \textbf{features}.
A \textbf{feature} is an independent, measurable property of an observed phenomenon.
Choosing which features of the training data yield the most \emph{relevant} information is a central problem in machine learning.
This is obvious from the perspective of supervised learning, but the same principle applies to RL problem environments.

Most succesful systems (prior to DQN) have relied on \textbf{handcrafted feature sets}, created by experts to capture the problem-relevant information so the RL system does not have to.
Other attempts have relied on \textbf{autoencoders} as a preprocessing step to lower the dimensionality of the observation before feeding it to the agent (used for example in \emph{neural Q-fitting} \cite{neural-q-fitted}).

This section introduces \emph{Mnih et al.}'s \textbf{DQN} \cite{atari-dqn}, a deep model-free learning method that uses \textbf{automatic feature extraction} (is capable of finding an efficient representation directly from interactions with the environment).
The researchers have demonstrated that a DQN agent with the same hyperparameters is able to learn from widely different environments, with no prior knowledge.

The paper benchmarks the agent on 49 video games for the Atari 2600, using the Atari Learning Environment \cite{ale-paper} (a extension layer over an emulator, specially designed to serve as a task environment).
The selection features diverse mechanics and some games require the development of long-term strategies to perform reasonably well at.
The vanilla DQN agent scored better than existing methods (including NEAT\footnotemark) on the vast majority of the games.
\footnotetext{standing for \emph{neuroevolution of augmenting topologies}, a genetic algorithm for generating neural network topologies}
Moreover, it managed a human-like performance at a considerable number of games, according to the evaluation techniques used in the paper.
According to this evaluation, the agent scored highest in \emph{Pong}, \emph{Enduro} and \emph{Breakout}, outperforming expert human players.

\subsubsection{Algorithm}
% Overview of the learning process: how we model the RL problem into a deep learning problem
The algorithm described in the paper is roughly based on \textbf{Q-learning} (described in Section \ref{rl:q-learning}), fitted with deep convolutional networks which approximate the action-value function $Q$, suggestively called \textbf{deep Q-networks}.

% Algorithm specificiation from paper
% Computing the gradients using Q-learning’s update formula
% Defining and optimizing the loss function
% RMSProp (see green paragraph above) 

% This should remain the end paragaph, in order to transition to explanations for experience replay and target networks.
Off-policy learning, bootstrapping (both characteristics of Q-learning) and functional approximation are \textbf{incompatible} at a fundamental level (the three elements comprise the so-called \emph{deadly triad} \cite{rlai}), as the combination fails to converge in theory.
In order to work properly, DQN modifies Q-learning in a few fundamental ways to overcome this incompatibility, by using \textbf{experience replay} and an additional \textbf{target network}.

\subsection{Policy Optimization Methods} \label{section:policy-opt}

\subsection{Actor-Critic Methods} \label{section:actor-critic}


\chapter{Playing Pac-Man using Deep Q-learning} \label{chapter:practical}
In this chapter, we focus on \emph{Pacman} as learning environment and on our collection of agent programs based on deep Q-learning.

The project itself consists of a light framework for training, running and evaluating RL agents.
Our framework is compatible with various other environments, as long as they respect the interface provided by OpenAI Gym.
The feature set includes automatic checkpoints, cloud-friendly deployment and a performance analysis toolkit.

In Section \ref{section:modelling-the-problem}, we justify why we choose to study the game of \emph{Pacman} and formalize the specifications of it as a learning environment.

In Section \ref{section:approach-algorithms}, we introduce some of the algorithms built into our collection of autonomous agents.
The main method is vanilla DQN, which was covered previously in Section \ref{section:dqn}.
However, in this section, we present two important improvements over it -- Double DQN and Prioritized Experience Replay.

In Section \ref{section:implementation}, we take a look at each subcomponent of an individual agent program and map each one to notions from our previous chapter.
Each agent has its own module and inherits a specific structure from a prototype.

In Section \ref{section:technologies}, we present a rundown of our technology stack, whose core is the Python programming language. The framework is built on the PyTorch machine learning library but makes use of many other ML and data science libraries.

We wrap up this chapter with Section \ref{section:user-manual} which consists of a short instruction manual for users to start local or remote training sessions, use the provided performance measurement tools and extend the existing collection of agents.

\clearpage

\section{Modelling the Problem} \label{section:modelling-the-problem}
The problem we solve in this thesis is three-fold.
Firstly, we specify a variant of a task environment for \emph{Pac-Man}, defining goals and rewards to view it through the lens of reinforcement learning.
Secondly, we train agents to ``solve’’ that environment, i.e. explore and learn to take optimal decisions to achieve the established goals.
And lastly, we compare agents based on their how close they were to their goals, what their mean reward was and how stable was their learning process.

\textbf{\textit{Pac-Man}} is a classic video game developed by Toru Iwatani and published by Namco Ltd. in 1980 \cite{pacman-in-academia}.
The game was originally released as an arcade game and it quickly became the most commercially sucessful arcade game ever created.
The original hit gave rise to a series of ``licensed clones'', which targeted a number of popular platforms, such as the Atari 2600, the Nintendo Entertainment System (NES), the Nintendo GameBoy, etc.
Besides its commercial success, \emph{Pac-Man} has been an appealing object of study for academia, across a number of fields, ranging from psychology to mathematics and computational intelligence \cite{pacman-in-academia}.

\begin{figure}[h]
    \centering
    \includegraphics[width=0.4\textwidth]{nes-pac-man.jpg}
    \caption{Screen capture of the starting configuration from Nintendo's NES variant of the game.}
    \label{fig:pac-man-screen}
\end{figure}

The game's mechanics are intuitive but engaging enough to be fun.
It allows simple four-directional movement.
The player is represented as a yellow character with a distinctive circular shape (with a ``missing slice''\footnotemark{} representing its mouth).
\footnotetext{Iwatani's stated inspiration for Pac-Man's design was a pizza with a missing slice. \cite{pacman-gameinformer}}
The board is a bidimensional maze, featuring tunnels which loop around each other, i.e. entering a tunnel on one side will transport the player out through the opposite side of the board.
\textbf{Pellets} (or \emph{dots}) are placed at almost every position, and the goal of every level is to collect all pellets.
Most renditions of the game feature four \textbf{ghosts} (faithful to the original), which will chase the player around the board.
Collisioning with any of the ghosts results in losing a life and being repositioned to the start (without losing progress on dots).
The ghosts have different ``personalities'' -- each of them has a different approach for chasing Pac-Man.
There are special power pellets in each of the four corners (seen in Figure \ref{fig:pac-man-screen}).
\textbf{Power pellets} grant Pac-Man temporary immunity from ghosts and changes the power dynamic in the game for a short time: the ghosts enter \textbf{``scared'' mode} and reverse their direction to run away.
In this mode, the player gets an increasing score for every ghost it eats.

\emph{Pac-Man} can be defined as a 2D gridworld environment, where exploration is necessary to reach the goal.
This class of task environments is common throughout RL literature and has some common properties:
\begin{itemize}
    \item they have discrete bidimensional representation;
    \item an agent has a finite set of options at every time step.
\end{itemize}

\textbf{Gridwords} are used in literature to clearly define and \textbf{isolate} agent goals.
A simplistic environment with a clear task is also beneficial for agent learning due to its low degree of environmental noise -- it does not ``distract'' the agent.
The ease of customizability of the rules inside a gridworld allows creating interesting strategy games to test agents’ decision-making capabilities.
New non-trivial grid-based environments are an area of open research -- for example, the \emph{Pommerman} benchmark \cite{pommerman-paper} has been formulated as a novel interesting multi-agent problem which does not yet have an optimal solution.

\emph{Pac-Man} has a few advantages which make it interesting to study from a RL perspective.

First of all, it provides a \textbf{shaped reward} built-into the game mechanics.
In reinforcement learning, a \textbf{sparse reward} is a large reward which is delayed for a relatively large number of steps.
A sparse reward would be, for example, if we would only reward the agent at the end of the game.
Shaped rewards, in contrast, equate with providing gradual feedback to the agent.
This can improve learning and lead to better outcomes earlier in training, as good behaviours become immediately obvious to the agent.

In \emph{Pac-Man}, the pellets, placed uniformly over the board, provide a shaped reward which directly reflects the goal of the game, i.e. to eat all pellets on a board without getting caught.
Moreover, they \textbf{incentivize exploration} as the reward is higher in places the agent has not visited before.

Second of all, \emph{Pac-Man} requires the agent to develop non-trivial strategies.
The game requires balancing two equally important subgoals: fully covering the game board to consume every available pellet, and running away from the chasing ghosts.
This gives rise to complex and potentially problematic situations for the player, which require \textbf{planning}.
This raises an interesting question regarding the potential of existing algorithms to learn \textbf{long-term strategies} similar in performance to human players.

The \textbf{reward function} we use in this thesis for our \emph{Pac-Man} environment is a combination of a few properties of the environment state.
The most important component is the \textbf{score}.
Score changes will determine a proportional reward for the agent.
Significant score increases are provided by eating pellets but a more interesting mechanic is eating ghosts when the player is invincible.
Each ghost eaten will provide a higher reward, thus conditioning the agent to chain as many ``kills'' as possible.
Typical of arcade games, the mechanics are built such that the score can \emph{only increase} over time.
This creates a necessity to \emph{counterbalance} by finding opportunities for \textbf{punishment}.
We choose to strongly penalize the the agent for losing a life.
There are other possible candidates for ``punishment'' events, but we consider ``death on ghost contact'' to be the most important for capturing the goal of the game.

\section{Algorithms} \label{section:approach-algorithms}
In this section, we present the main algorithms powering the autonomous agents evaluated in this thesis. In our collection, \textbf{vanilla DQN} serves both as a stand-alone implementation, as well as a fundamental building block of more advanced algorithms.
The original DQN, however, is already covered by Section \ref{section:dqn} of our theory chapter and we will omit it here for conciseness.
Instead, this section deals with \textbf{double DQN} and \textbf{prioritized experience replay (PER)}, both of which address shortcomings of the original algorithm.

\subsection{Double Q-learning}
Double DQN \cite{ddqn-paper} is an algorithm in the deep Q-learning family, discovered and published by DeepMind researcher Hado van Hasselt along with several other colleagues.
The study determines that general Q-learning (and, by extension, deep Q-learning) suffers from a problem known as a \textbf{maximization bias} and adapts the original algorithm to account for this and counteract the bias.

Stated simply, a \textbf{maximization bias} is an implicit preference of the algorithm to overestimate values, despite divergence from the true value.
Some control algorithms, such as Q-learning, fundamentally depend on using the maximum over estimated values as an estimate for the maximum value \cite{rlai}.
This leads to situations where noisy reward signals, prevalent in real-world environments, can significantly destabilize learning and slow down training progress.

Hasselt proposes \emph{double learning} to counteract this bias.
This requires two independent approximators of the action-value function -- we will denote them $Q_1$ and $Q_2$.
At every step, the approximators are \emph{intermittently} used either to pick an action or to estimate its value.
The paper proves that this decoupling is sufficient to accomplish an unbiased estimation.
Which function serves which role can be chosen uniformly at random at every learning step at practically no cost.
Equation \ref{eqn:double-learning-rule} states the double learning \textbf{update rule}, considering the situation in which $Q_1$ picks the action (and is updated) while $Q_2$ gives the estimate which we bootstrap from.

\begin{equation} \label{eqn:double-learning-rule}
\begin{aligned}
Q_1(S_t, A_t) & \leftarrow Q_1(S_t, A_t) \\
    & + \alpha [ R_{t+1} + \gamma Q_2(S_{t+1}, \argmax_{a \in A} Q_1(S_{t+1}, a)) - Q_1(S_t, A_t)]
\end{aligned}
\end{equation}

A drawback of the double learning algorithm is that it requires double the amount of memory of the original. 
Despite this, it still performs one update per step, so the running time is equivalent to that of its predecessor.
% to steal more content space, include and explain the example from RLAI

\subsection{Prioritized Experience Replay (PER)}
\textbf{Prioritized experience replay} (often shortened as \emph{PER}) is an extension of deep Q-learning, also developed by a team at DeepMind, led by Tom Schaul \cite{per-paper}.
The paper focuses on improving experience replay from classic DQN.

Recall from Section \ref{section:dqn} the principle of \textbf{experience replay}, which consists of storing all transitions the agent experiences in order to revisit them later during training.
Reusing experiences over multiple update steps reduces the number of samples required for learning.
The original approach suggests that the experiences are revisited by sampling them uniformly at random from the replay memory.
This has the important role of \emph{decoupling} experiences -- which in the case of successive video game frames, are by nature highly correlated -- to avoid introducing \emph{high variance} into the system (which can cause overfitting).

The foundational hypothesis of the paper at \cite{per-paper} is that some transitions have higher ``teaching’’ potential than others.
The \emph{prioritization} mentioned in the title refers to finding an ordering over experiences which properly reflects this potential.
The authors chose \emph{absolute TD-error} as a good approximation for utility in learning -- for several reasons.
Firstly, since larger errors suggests that a prediction is further from the agent’s current expectations, minimizing larger error would naturally improve 
This supposition has intuitive appeal because \textbf{humans} likewise learn proportionally more from experiences that contradict their existing beliefs the most.
Additionally, TD-error is already computed as part of the normal update cycle in DQN so extending the implementation to PER does not require significant overhaul of the algorithm.

The paper develops two separate models for prioritizing experience based on error.
Each model specifies how priorities are computed and stored.

\textbf{Rank-based prioritization} involves keeping track of the rank each transition occupies.
Each transition would thus have an assigned priority of $p = 1 / rank$.
This requires an additional data structure, capable of online sorting, such as a heap, as some ranks will change at every update step.

Since the replay memory can potentially store millions of transitions, \textbf{optimization} is necessary to prevent severely hindering performance.
Changing every rank at every update step is impractical and would outweigh the benefits of prioritization.
In order to mitigate this, the authors propose to only re-evaluate the ranks of transitions in the sampled mini-batch.
This \textbf{``lazy’’ approach} retains its desirable properties of learning more efficiently than DQN while not adding an insane amount of computation.

\textbf{Proportional prioritization} simply assigns a priority equal to the absolute TD-error plus an $\epsilon$: $p_{i} = |\delta_{i}| + \epsilon$.
The $\epsilon$ (which should not be confused with our exploration coefficient in $\epsilon$-greedy) is an offset keeping all priorities above $0$.

The second part of the algorithm is common to both models and consists of actually assigning probabilities based on priority and building the distribution over the transitions. The formula in (\ref{eqn:per-probability}) gives an outline of the method.

\begin{equation} \label{eqn:per-probability}
    P(i) =  \left( \frac{p_i}{\sum_{k}{p_{k}}} \right)^\alpha
\end{equation}

The $\alpha$ parameter in this equation represents the degree of prioritization, i.e. a higher $\alpha$ will determine a larger difference between the probabilities of any two priorities.

To sample according to the obtained priority, a sum tree is used.
A \textbf{sum tree} is a type of segment tree -- a tree data structure which stores statistics (minimum, maximum, sum etc.) of its leaf nodes (which also represent an array).
The structure is designed to compute interval queries, such as the sum between indices $x$ an $y$, in $\mathcal{O}(\log{n})$, while also allowing fast updates to the underlying data.

\section{Implementation} \label{section:implementation}
% show a conceptual overview of each high-level moving part
% and how they communicate
% Trainer class as an entry point
% show the class diagram and explain more low-level details}

\section{Technologies} \label{section:technologies}
% PyTorch, Google Cloud Compute, OpenAI Gym, Jupyter Lab
% In the future, TensorBoard

\section{User Manual} \label{section:user-manual}
% Commands for training, running, deploying, flags and configuration variables

\chapter{Results and Evaluation} \label{chapter:results}
This chapter introduces our methodology, presents our experiment setup and results, and compares them to the expectations set by existing benchmarks.
First of all, we introduce and justify the methods we use to evaluate agent performance in the Pac-Man environment.
We then show the results obtained using the chosen technique and analyze them, using the tools provided in our framework.
We comment on those results and compare them to what other similar studies have shown.

\section*{Evaluation in Reinforcement Learning}
\textbf{Evaluation} is a fundamental part of any machine learning project and reinforcement learning is not an exception to this rule.
The goal of evaluation in RL is compatible with the goal of its encompassing field and is analogous to that of supervised learning, in the sense that both are concerned with measuring how well a model \emph{generalizes}.

The aim of evaluation in RL is to measure the extent to which the acquired behaviour of the agent generalizes to the class of environment it learned in.
In other words, researchers try to certify if the agent learns meaningful \emph{skills} -- patterns in behaviour which are general enough to solve the problems they are developed to solve, even if presented from a new perspective.

In a similar manner, in supervised learning, models are evaluated based on their ability to accurately predict, estimate or classify previously unseen data -- their generalization ability.
A key difference between RL and its supervised counterpart is the latter's clearer methodology of dividing the dataset.
Experiments in supervised learning can and must split their data into specific subsets -- one for training and one for evaluation. In the RL setting, this separation is seldom possible.

Attempts to correct this overlap between training and evaluation in experimental RL is an active area of research.
One compelling paper concerning itself with this problem, published by OpenAI researchers, proposes building \emph{procedural generation} into environments \cite{procgen-paper}.

\section*{Setup and Methodology}
The evaluation \textbf{methodology} we chose for this thesis was used before across numerous RL studies, including in DeepMind's DQN and Double DQN works \cite{atari-dqn,ddqn-paper} and has been originally proposed in the Atari Learning Environment (ALE) paper \cite{ale-paper}, which was specifically developed to analyze agent perfromance in video game environments on the Atari 2600.
We follow the above guidelines closely, with a few exceptions.
For example, we will not be including including human performance in our benchmark.

% REFACTOR THIS --------------------------------------------------------------
% To prevent overfitting, the environment has \textbf{added stochasticity}.
% A significant portion of video game environments are deterministic by nature, including Pac-Man.
% If the agent were to replay the same sequence of moves over multiple gameplays, the environment would respond in the exact same way.
% The problem is that an ``imposter’’ agent could overfit and memorize the level over time, resulting in an agent that can solve the level reasonably efficiently, but which fails if we tweak even fine details.
% This is prevented by the stochasticity of the environment and the stochasticity of its own model.
% In order to ensure that no two training runs are the same from an environmental stand-point, the agent’s choice is replayed in the environment.
% This can be a fixed or a random number of times, which we denote $k$.
% Some task environments choose a $k$ in $[1, 3]$.
% We use the fixed variant\footnotemark{} of this technique, which has been proposed and used by the DeepMind team in the original DQN study \cite{atari-dqn}.
% \footnotetext{DQN uses $k = 4$ for most games, except for Space Invaders which requires $k = 3$ because of an overlap with the period at which the projectiles blink \cite{atari-dqn}.}

Additionally, we use \textbf{epsilon annealing} for the $\varepsilon$-greedy policies, another technique in the DQN standard.
The approach consists of linearly decreasing epsilon from $1$ to a predetermined lower bound, usually $10\%$ or $5\%$, to ensure non-zero exploration probability over the entire training run.
The role of this is to ensure that the agent does not get stuck in a local optimum, as well as to prevent overfitting to the environment.
If we allow the exploration coefficient to go all the way to zero, we risk having an agent learn a pattern that is inefficient or even counterproductive and would have no way of stopping it short of restarting the training run.

Our experiment setup starts from well-established standards in order to find ways to tweak them.
The implementation we use for our task environment is \texttt{MsPacman-v4}, provided by OpenAI within the \texttt{gym} package.
The base name of the game is \texttt{MsPacman}, designating the Atari 2600 version of Pac-Man.
The package offers multiple variables which control stochasticity of the environment.
An environment labeled \texttt{Deterministic} will have a fixed frameskip, set to $k = 4$ in our case.
The label \texttt{v4} indicates that the agent action is not repeated (besides the frame-skipping aspect), in contrast to \texttt{v0} which specifies a 25\% chance the previous action is repeated.

Our choice of hyperparameters is illustrated in Table \ref{tab:our-hyperparameters}.

\begin{table}
    \centering
        \begin{tabular}{ll}
        Variable                           & Value   \\
        learning rate $\alpha$             & 1e-4    \\
        discount factor $\gamma$           & 0.99    \\
        initial $\varepsilon$              & 1       \\
        terminal $\varepsilon$             & 0.05    \\
        $\varepsilon$ decay                & 8.62e-5 \\
        batch size                         & 32      \\
        swap frequency (number of steps)   & 1000    \\
        PER $\alpha$                       & 0.6     \\
        PER $\epsilon$                     & 0.01
        \end{tabular}%
    \caption{Hyperparameters used to train our agent models.}
    \label{tab:our-hyperparameters}
    \end{table}

% gym
% hyperparameters

\section*{Results}
% plot some graphs
% write the conclusion
% read Roman Ring's thesis again

\chapter{Conclusion and Future Work} \label{chapter:conclusion}
This thesis began by presenting the fundamentals of Reinforcement Learning (RL) -- a field of machine learning concerned with developing systems which learn based on interaction with the environment.
What distinguishes RL from other methods is the concept of reward signal, which provides feedback on whether the system behaves in a desirable way with regard to the goal of a task.

We furthermore explore how the transfer of knowledge from deep learning to the classical methods in RL is revolutionizing the field.
In the second part of Chapter \ref{chapter:background}, we presented some of the techniques that emerged to build more performant intelligent agents, leveraging the strength and robustness of neural networks, and especially of the ConvNet architecture.

In order to demonstrate the principles of deep RL in action, we have implemented a light framework, featuring a collection of algorithms including Deep Q-learning, originally described in \cite{atari-dqn}, and its successors -- Double DQN, Prioritized Experience Replay (PER) and Double PER.
We benchmarked and analyzed the performance of our selected approaches on our task environment of choice -- Pac-Man. Solving Pac-Man is an interesting task and features some open challenges, as mentioned in Section \ref{section:modelling-the-problem}.
Our ranking shows the top performer is PER both in learning and in evaluation, despite Double DQN coming close and exhibiting a more stable ascent.

The following paragraphs give a summary of the most notable \emph{areas of potential improvement} for this work.

One possible extension to this thesis is hyperparemeter search.
\textbf{Hyperparameter tuning} or \textbf{search} is a problem in machine learning, which consists of tuning the set of hyperparameters to maximize the usefulness of the learning approach \cite{hyperparam-search-paper}.
Most studies have relied on the entire set of 49 Atari games to study the efficiency of their approaches.
In our thesis, however, we solve a more precisely defined task and could have benefited from tuning hyperparameters to some extent.
There is a strong probability that the set of hyperparameters which worked the best on Pong or Enduro does not work as well in Pac-Man.
However, one issue with auto-tuning is its high computational demand, in that applying it to a relatively large problem such as ours requires a large amount of continuously-available computational resources.
On the basis of these large computation costs, we have decided that it would be cost ineffective to pursue.

Another such extension point is including policy-based and actor-critic methods into the benchmark.
In our background section, we touch on the subject of policy-based methods and their particular advantages.
The problem with policy-based RL is the novelty and increased complexity of the algorithms, over the more well-established techniques.
In other words, the state-of-the-art is not accessible to entry-level enthusiasts and would require far more work to implement from scratch, test and deploy.

To sum up, we have seen that autonomous RL agents can learn useful behaviour.
However, the study of autonomously learning RL agents in games has been challenging.
As we have seen with DQN and its successors, the method's greatest weakness remains sample efficiency -- agents need to train for millions of steps before they demonstrate any significant amount of ability.
And even then -- as we pay closer attention to generalization in deep RL -- we find that most Q-learning-based methods only manifest a somewhat limited ability to generalize, outside of very simple strategies (such as Pong, the game with the highest score obtained by DQN \cite{atari-dqn}).

With this in mind, the road ahead looks bright for RL, with a recently released paper from DeepMind – \emph{Agent57} (2020) \cite{agent57-paper}. We note that the latest research trends seem to be using mixed approaches: DQN and policy optimization algorithms, in an attempt to break the barriers set by their predecessors.
\emph{Agent57} has managed to break every record on the Atari benchmark and has obtained scores above the human baseline on 57 games in the set.

\bibliography{references}

\end{document}